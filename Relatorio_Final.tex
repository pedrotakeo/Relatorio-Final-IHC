\documentclass{article}
\usepackage[portuguese]{babel}
\usepackage{graphicx} 
\usepackage{longtable}

\title{Relatório Final IHC}
\author{
    Beatriz Pontes Camargo\\
    \footnotesize (GRR20242966)\\[1em]
    Eduardo Kaluf\\
    \footnotesize (GRR202421770)\\[1em]
    Giuliano Tavares\\
    \footnotesize (GRR202420305)\\[1em]
    \and 
    Pedro Takeo Shima\\
    \footnotesize (GRR20240627)\\[1em]
    Sergio Sivonei de Sant'Ana Filho\\
    \footnotesize (GRR20242337)\\[1em]
    Thamiris Yamate Fischer\\
    \footnotesize (GRR20243604)\\[1em]
}


\begin{document}

\maketitle
\clearpage
\tableofcontents
\clearpage

\section{Ficha de Identificação}

O sistema avaliado foi o aplicativo SOUFPR, que visa centralizar e facilitar o acesso à informação, serviços e contato para/com a UFPR.
A aplicação tem como Stakeholders centrais os estudantes, mesmo que a cadeia de Stakeholders seja mais extensa. 

Nome completo de cada integrante bem como a técnica utilizada:
\begin{itemize}
    \item Beatriz Pontes Camargo - SUS
    \item Eduardo Kaluf - Oito Regras de Ouro
    \item Giuliano Tavares - MIS - 
    \item Pedro Takeo Shima - Leis da Simplicidade
    \item Sergio Sivonei de Sant'Ana Filho - Percurso Cognitivo 
    \item Thamiris Yamate Fischer - Design Universal
\end{itemize}

O total de problemas únicos encontrados foi de 32 problemas distintos.
Cada técnica teve 10 problemas encontrados, com exceção do SUS, pela avaliadora Beatriz P. Camargo que identificou 9.
Já sobre a quantidade de problemas por severidade, temos a seguinte distribuição:
\begin{itemize}
    \item Gravidade 4 (Problema catastrófico): 0
    \item Gravidade 3 (Problema Grande/Grave): 8
    \item Gravidade 2 (Problema médio): 13
    \item Gravidade 1 (Problema cosmético superficial): 10
    \item Gravidade 0 (não necessariamente um problema): 1
\end{itemize}

Quanto ao total de problemas encontrados por um único avaliador, temos a mesma distribuição que por técnica, onde a avaliadora Beatriz Pontes Camargo encontrou 9 e o restante da equipe encontrou 10.

\section{Matriz de avaliação}

Ao realizar a avaliação de um sistema específico, visando tanto a identificação de minúcias não percebidas durante o projeto quanto a priorização de correções, é possível afirmar que a organização metodológica é crucial nestas atividades.

Ao realizar uma avaliação individualmente, isto já se torna ponto-chave do trabalho, porém, ao executar um revisão do sistema interativo em grupo, a sistematização solidifica-se como a base de tudo que será feito. Sendo assim, uma técnica comum e eficaz é elaborar uma grande matriz com todos os problemas encontrados, fornecendo apenas uma visão geral sobre o que foi encontrado nas inspeções, a fim de, posteriormente, discutir e opinar sobre os itens registrados.

No caso da equipe Shima, a própria elaboração da matriz foi feita em conjunto, na qual problemas repetidos acabaram agrupados de maneira concisa, e o destaque em amarelo foi fornecido aos avaliadores que os identificaram, dando um primeiro indício dos problemas que saltam mais aos olhos na plataforma.

Feito isso, as severidades já existentes passaram a ser registradas e quem não teve sucesso em localizar determinado problema em sua própria verificação, obteve a possibilidade de refletir sozinho sobre o quão grave achava que cada problema era. Por fim, houve uma discussão em grupo, determinando a severidade final dos problemas encontrados. Durante a conversa, cada um expôs seu ponto de vista, deixando explícito o porquê de determinada severidade ter sido registrada, maneiras de contornar aquele problema, onde ele foi localizado e detalhes mais aprofundados.

É importante ressaltar que na matriz de avaliação, a severidade final não é uma média, mas sim um consenso. Observa-se que alguns problemas possuem severidade final "3", mesmo com um ou mais membros da equipe tendo classificado-o como severidade 1. Quando isso acontece é porque, durante o processo de conversa, esses membros repensaram a influência que tal problema possui sobre a aplicação; o mesmo ocorre quando severidades individuais são maiores do que a severidade final.

Segue abaixo a matriz completa após a discussão e elaboração das descrições resumidas:

\section{Descrição dos problemas, severidade e sugestões de melhoria}

As severidades e as sugestões de melhoria podem ser encontradas na seção ~\ref{tab:app:avCons}, contendo a tabela de avaliação consolidada. A avaliação consolidada está separada em duas tabelas: a Tabela ~\ref{tab:app:avCons1}, contendo o local onde ocorre o problema, uma descrição sucinta do problema e a severidade final decidida pelo grupo, e a Tabela ~\ref{tab:app:avCons2}, apresentando as sugestões de correções e melhorias.

\subsection{Descrição dos Problemas}

Esta seção detalha as descrições dos problemas e explicita as formas de reproduzi-los. Durante o 
processo de avaliação do aplicativo \textbf{SouUFPR}, houve uma atualização, resolvendo alguns dos problemas citados. Sendo assim, para tais problemas, será indicado caso tenha sido resolvido e haverá uma explicação de como poderia ser reproduzido antes da atualização. Os problemas estão numerados de acordo com seus IDs.

\begin{enumerate}
    \item O fluxo de telas apresenta alguns problemas relacionados a navegação complicada devido a quantidade excessiva de telas intermediárias. Essas telas explicitam informações já citadas anteriormente, ou, se apresentam alguma informação nova, podem ser implementadas em outras telas, de forma a evitar redundância. Algumas dessas páginas intermediárias foram removidas ná nova atualização, porém, entrando na página de feedbacks do aplicativo, ainda temos uma tela que poderia ser removida, levando o usuário diretamente à página de sugestões.
    \item O aplicativo apresenta uma quantidade excessiva de redirecionamentos para sites externos. Esses redirecionamentos ocorrem em grande parte para funcionalidades que poderia ter sido desenvolvidas de forma nativa.
    \item 
    \item 
    \item 
    \item 
    \item 
    \item 
    \item 
    \item 
    \item 
    \item 
    \item 
    \item 
    \item 
    \item 
    \item 
    \item 
    
\end{enumerate}
    

\section{Apreciação geral da qualidade do sistema com base nos resultados das avaliações}

Após a etapa de avaliações individuais, nas quais cada avaliador aplicou seu respectivo método e as discussões para unir cada perspectiva, podemos apresentar a apreciação a respeito da qualidade geral do sistema SouUFPR.

Um primeiro aspecto analisado e fortemente pontuado pelos avaliadores foi a Acessibilidade. Em geral, o sistema é acessível e isso é reforçado principalmente pelo uso de ferramentas que avaliam esse aspecto, como o LightHouse e o WAVE, conseguindo uma acurácia boa no LightHouse e havendo poucos problemas aparentes no WAVE. Todavia, apesar de uma boa avaliação nessas extensões, há aspectos graves que dificultam o acesso para pessoas que necessitam de leitores de tela, como redundância na descrição alternativa de ícones e a ausência de uma main landmark.

Outros aspectos observados foram a lógica confusa, a frustração de uso do sistema e o fluxo de telas complicado e exagerado. Os avaliadores, em geral, apontaram que há várias telas intermediárias até chegar a um objetivo final, coisa que poderia ser mais simples e curta, indo direto ao objetivo, por exemplo. Também que há ícones e partes que causam confusão, além de induzir o usuário a uma determinada ação ou expectativa.

Levando em consideração a comunicabilidade e o feedback para o usuário, ou seja, como que o SouUFPR interage com o usuário e dá feedbacks. Nesse ponto, foi detectado problemas com partes redundantes ou pouco visíveis do site, por exemplo, botões pouco aparentes e baixo contraste. 

Apesar dos pontos negativos, é importante ressaltar que o sistema avaliado não se consolida como uma aplicação inutilizável no que diz respeito a um usuário que não dependa de recursos de acessibilidade (como os landmarks citados). O aplicativo é de fácil aprendizagem e possuí uma linguagem simples, pontos positivos de sua implementação. Não obstante, o fato de estar em fase inicial é um bom indicativo quanto a resolução futura dos problemas encontrados.

Em suma, o sistema SouUFPR pode ter a sua qualidade avaliada em três pilares que pontuam aspectos a melhorar da aplicação que são: Acessibilidade, Fluxo de Telas e Confusão do Usuário, e Comunicabilidade e Feedback ao Usuário. Todavia vale ponderar que a aplicação também possui pontos positivos que a descrevem, como citados anteriormente. Logo, é possível dizer que o aplicativo é simples, com aprendizagem rápida e visa cumprir seu principal objetivo, ou seja, ser um aplicativo direcionado para a comunidade acadêmica da UFPR que centraliza informações importantes para o cotidiano dessa. Entretanto, há alguns pontos de melhoria como: melhorar a acessibilidade para pessoas que necessitam de leitores de tela, ter um fluxo de telas mais intuitivo, simples e fluído, deixar os ícones com significados mais explícitos e evidenciar melhor algumas funcionalidades, como por exemplo, alguns dos botões.


\section{Apreciação sobre os resultados da avaliação}

Conforme apresentado no tópico 1, foi possível observar uma convergência na quantidade dos problemas encontrados, onde praticamente todos os avaliadores identificaram ao todo 10 problemas cada, mesmo aplicando técnicas distintas. Essa uniformidade pode ser atribuída ao estágio atual do sistema avaliado, que ainda possui algumas funcionalidades limitadas ou em desenvolvimento.

Embora a soma bruta de problemas encontrados pela equipe seja de 59, durante a consolidação na matriz de avaliação esse número foi reduzido para 32 problemas únicos e, posteriormente, na consolidação final a equipe decidiu selecionar os 18 problemas mais pertinentes. Essa redução drástica entre o bruto e os incluídos na matriz evidencia a forte sobreposição de diagnósticos entre as diferentes técnicas. Além disso, notou-se nos relatórios individuais que alguns avaliadores filtraram problemas que não se enquadraram nos critérios específicos da técnica utilizada.

Quanto à eficácia individual na consolidação final, as técnicas SUS (Beatriz) e Design Universal (Thamiris) se destacaram, pois ambas identificaram a maior quantidade de problemas quando estes foram agrupados e condensados, foram 15 ao todo. Além disso, o Design Universal foi a técnica que conseguiu identificar a maior quantidade de problemas exclusivos, 2 ao todo. Esse resultado superior de ambas as técnicas pode ser justificado pela abrangência de seus escopos, enquanto o SUS avalia a percepção global de usabilidade, o Design Universal verifica barreiras de acesso transversais a todo o sistema, permitindo apontar falhas que heurísticas mais focadas em tarefas específicas acabam ignorando. Vale ressaltar que a diferença entre o número dito inicialmente de 10 por pessoa e o número atual acontece devido ao agrupamento e condensamento já citado, onde certos problemas apontados pelos avaliadores na avaliação individual foram colocados em mais de um durante a consolidação final por se encaixarem em mais de um grupo.

Além disso, os custos das técnicas aplicadas são justificáveis e tais resultados não seriam possíveis caso houvesse redução no número de métodos ou avaliadores. Isso ocorre porque, embora exista sinergia entre algumas abordagens (como o MIS e as Leis da Simplicidade, que focam no nível mental e abstrato), as seis técnicas operam, no geral, sobre cenários distintos. Essa variedade permitiu extrair uma gama muito mais ampla de problemas, resultando em um diagnóstico rico que abrange desde a compreensão de ícones até a acessibilidade para grupos diversos, fundamentando assim um plano de ação muito mais completo. Em um grupo reduzido até seria possível chegar a um resultado semelhante, mas teria que tomar um cuidado maior para não escolher métodos que foquem no mesmo objetivo.

Uma consideração relevante sobre o processo de avaliação da equipe foi o rigor metodológico, o que pode ter impedido de certos problemas irem para a avaliação individual, pois desde o início foi definido qual técnica seria usada por cada avaliador e conforme relatado em alguns relatórios, isso filtrou problemas, pois alguns problemas encontrados não se encaixavam na técnica aplicada. Visto isso, para uma futura avaliação, seria importante reportar todos os problemas encontrados, independentemente da técnica, deixando a parte de classificar para a etapa de consolidação. Assim é possível ter uma visão muito mais rica em detalhes do sistema avaliado.

Por fim, é importante destacar que todos as técnicas aplicadas pela equipe foram de inspeção, ou seja, foram técnicas que durante a avaliação não teve a presença de um usuário real, todas as técnicas simulavam um usuário. Portanto, caso fosse possível, uma próxima etapa para aprimoramento do sistema seria teste de usabilidade com usuários reais, pois seria possível ver na prática se os 18 problemas pertinentes decididos pela equipe realmente refletem a realidade dos stakeholders do souUFPR. 

\clearpage









\appendix

\section*{APÊNDICES}

\section{Matriz de Avaliação}

\begin{longtable}{l p{0.6\textwidth} p{0.3\textwidth}}
    \centering 
    \textbf{ID} & \textbf{Problema} & \textbf{Sev} \\\hline

    
        \#01 & fluxo de telas muito complicado. telas intermediárias que se tornam redundantes & 3. Problema grande ou grave \\\hline
        \#02 & horários do intercampi são mostrados simultaneamente e não apresentam alguma forma de filtro & 3. Problema grande ou grave \\\hline
        \#03 & Muitas descrições alternativas são idênticas no site, apesar dos itens serem completamente diferentes & 3. Problema grande ou grave \\\hline
        \#04 & ícones na barra inferior não possuem legenda (ex: casa, intergração, envelope, pessoa). O usuário precisa clicar no ícone para identificar o que este representa. & 3. Problema grande ou grave \\\hline
        \#05 & Descrição dos ícones é a mesma "icone" havendo redundância. Pode gerar dúvidas para quem utiliza leitores de tela & 3. Problema grande ou grave \\\hline
        \#06 & Para acessar o vídeo de como preencher um comentário de sugestão é exaustivo e pouco óbvio. Ou seja, o link do vídeo, para um usuário desatento é pouco perceptível, além de demandar grande esforço de procurar e aprender um tutorial para poder deixar uma sugestão & 3. Problema grande ou grave \\\hline
        \#07 & A página não tem um main landmark – ou seja, não existe um ponto arquitetônico definido que represente o conteúdo principal da página, sendo um problema para a identificação das partes do site pelos leitores de tela & 3. Problema grande ou grave \\\hline
        \#08 & Redirecionamento excessivo para features que podiam ser implementadas internamente & 3. Problema grande ou grave \\\hline
        \#09 & páginas que precisam de login do usuário precisam passar por duas páginas antes de realmente levar para o login. & 2. Problema médio \\\hline
        \#10 & algumas páginas dedicadas a guião não contem nenhuma informação & 2. Problema médio \\\hline
        \#11 & botões formatados como links. Deixa o layout confuso e simplifica demais o layout, levando o usuário a pensar que os botões são somente texto não reativo & 2. Problema médio \\\hline
        \#12 & símbolos do cardápio (legenda) não indicam significado de forma explícita & 2. Problema médio \\\hline
        \#13 & É necessário rolagem completa da página para achar o botão voltar nos horários do intercampi. Ações utilizadas frequentemente devem estar facilmente visíveis & 2. Problema médio \\\hline
        \#14 & Não existe a opção de costumizar o site para modo escuro/claro & 2. Problema médio \\\hline
        \#15 & Notícias desatualizadas e redirecionamento para página do Instagram & 2. Problema médio \\\hline
        \#16 & O usuário é deslogado imediatamente ao sair da aplicação & 2. Problema médio \\\hline
        \#17 & Falta de botão de "voltar" dentro das seções internas, forçando uso do navegador & 2. Problema médio \\\hline
        \#18 & Textos de manchetes (ex: "Curitiba 332 anos!") têm contraste baixo com o fundo da imagem & 2. Problema médio \\\hline
        \#19 & Quando se acessa a localização de algo, ele redireciona para o google maps, porém usa coordenadas ao invés de endereço. & 2. Problema médio \\\hline
        \#20 & Em "minha conta", as informações aparecem em campos que parecem editáveis mesmo sem ser. & 2. Problema médio \\\hline
        \#21 & Em Ajuda, as opções de Carteirinha e Configurações podem causar confusão por parecerem abas de interação com o sistema. Carteirinha pode parecer que vai redirecionar para carteirinha (como na home) e Configurações para uma página de configurações do Aplicativo (cuja qual nem existe). & 2. Problema médio \\\hline
        \#22 & Após fazer o login, o usuário é sempre redirecionado para a tela home e não a tela que queria acessar & 1. Problema cosmético ou superficial \\\hline
        \#23 & ícones de features muito similares de forma repetida na home screen & 1. Problema cosmético ou superficial \\\hline
        \#24 & sistema de grid para escolha de campi e cardápio parece muito complicado devido a quantidade de itens & 1. Problema cosmético ou superficial \\\hline
        \#25 & ícones desproporcionalmente grandes. Ao aplicar zoom o hud é mal comportado, não mantendo uma mesmo padrão ou se adaptando para algo que faça mais sentido para a aplicação. & 1. Problema cosmético ou superficial \\\hline
        \#26 & Movimentação através do site somente pelo teclado fica difícil pela cor azulada da maior parte dos botões, dessa forma o highlight não é facilmente visível e quase impossível saber o que está selecionado & 1. Problema cosmético ou superficial \\\hline
        \#27 & ícones de ajuda (ex: RU, Intercampi, Campus) estão bem distribuídos, mas não há agrupamento visual ou hierarquia & 1. Problema cosmético ou superficial \\\hline
        \#28 & botões de voltar a página anterior não estão no padrão conforme outros botões, facilmente confundíveis com texto simples na interface. & 1. Problema cosmético ou superficial \\\hline
        \#29 & Aba Sugestões o texto é ruim de ler e abre em outra página com instruções um pouco confusas & 1. Problema cosmético ou superficial \\\hline
        \#30 & Na seção ajuda a qual possui uma grade de ícone mostrando os tópicos de ações disponíveis, o design feito foi usando um ícone para simbolizar o tópico e abaixo do ícone um texto dizendo o nome do tópico, porém ao entrar em algum dos tópicos o design muda e é optado apenas por deixar o título, abandonando o ícone & 1. Problema cosmético ou superficial \\\hline
        \#31 & A galeria de fotos estão com fotos dos campi muito grandes, ocupando um espaço desnecessário na tela e sendo necessário mais 'scrolling' para ver os outros campi da universidade, dificultando a comunicabilidade. Além do mais, o nome dos campi estão pequenos em comparação com as fotos, levando o usuário a ficar um pouco perdido durante o primeiro acesso caso não conheça o campus olhando apenas a foto. Outro problema é a qualidade das fotos está com uma resolução baixa. & 1. Problema cosmético ou superficial \\\hline
        \#32 & falta de informações que seriam relevantes no contexto do aplicativo, como localização de bibliotecas e diferentes setores. & 0. Não é necessariamente um problema \\\hline
    
\end{longtable}
\clearpage








% ----------------TABELA AVALIACAO CONSOLIDADA PT1


\section{Avaliação Consolidada}
\label{tab:app:avCons}

\subsection{Tabela 1: Identificação e Severidade dos Problemas}
\label{tab:app:avCons1}

\begin{longtable}{p{0.5cm} p{0.18\textwidth} p{0.50\textwidth} p{0.17\textwidth}} 

    
    \renewcommand{\arraystretch}{1.5} 

    \textbf{ID} & \textbf{Local onde Ocorre} & \textbf{Descrição do Problema} & \textbf{Severidade Final} \\\hline

    \#1 & Páginas que precisam de login (carteirinha, minha conta, intercampi, etc.) & Fluxo de telas muito complicado. As telas intermediárias que se tornam redundantes & 3. Problema grande ou grave \\\hline
    \#2 & Várias páginas (login, sugestões, campi, etc.) & Redirecionamento excessivo para features que podiam ser implementadas internamente & 3. Problema grande ou grave \\\hline
    \#3 & Página de horários do intercampi, grid de escolha de campi e cardápio & Falta de opção de filtro para páginas que mostram muitas informações & 3. Problema grande ou grave \\\hline
    \#4 & Várias páginas & Descrições alternativas (alt) redundantes e muito genéricas para ícones e imagens & 3. Problema grande ou grave \\\hline
    \#5 & Navbar (independe de páginas) & ícones na barra inferior não possuem legenda (ex: casa, interrogação, envelope, pessoa). O usuário precisa clicar no ícone para identificar o que este representa. & 3. Problema grande ou grave \\\hline
    \#6 & Todas as páginas & A página não tem um main landmark – ou seja, não existe um ponto semanticamente definido que represente o conteúdo principal da página, sendo um problema para a identificação das partes do site pelos leitores de tela & 3. Problema grande ou grave \\\hline
    \#7 & Página da Seção de Ajuda & Algumas páginas dedicadas a guiar o usuário não contem nenhuma informação & 2. Problema médio \\\hline
    \#8 & Tela intermediária de login/ tela de escolha de ônibus do intercampi & botões formatados como links. Deixa o layout confuso e simplifica demais o layout, levando o usuário a pensar que os botões são somente texto não reativo & 2. Problema médio \\\hline
    \#9 & Cardápio do RU & Símbolos do cardápio (legenda) não indicam significado de forma explícita & 2. Problema médio \\\hline
    \#10 & Várias páginas & Falta de botões e opções facilmente acessíveis para desfazer ações do usuário & 2. Problema médio \\\hline
    \#11 & Todo o aplicativo & Não existe a opção de costumizar o site para modo escuro/claro & 2. Problema médio \\\hline
    \#12 & Independe de páginas & O usuário é deslogado imediatamente ao sair da aplicação & 2. Problema médio \\\hline
    \#13 & Várias páginas & Textos em algumas páginas têm contraste baixo com o fundo & 2. Problema médio \\\hline
    \#14 & Página de informações dos campi & Quando se acessa a localização de algo, ele redireciona para o google maps, porém usa coordenadas ao invés de endereço. & 2. Problema médio \\\hline
    \#15 & Página de ajuda & Em Ajuda, as opções de Carteirinha e Configurações podem causar confusão por parecerem abas de interação com o sistema. Carteirinha pode parecer que vai redirecionar para carteirinha (como na home) e Configurações para uma página de configurações do Aplicativo (cuja qual nem existe). & 2. Problema médio \\\hline
    \#16 & Telas de login/ home & Após fazer o login, o usuário é sempre redirecionado para a tela home e não a tela que queria acessar & 1. Problema cosmético ou superficial \\\hline
    \#17 & Várias páginas & ícones desproporcionalmente grandes. Ao aplicar zoom o hud é mal comportado, não mantendo um mesmo padrão ou se adaptando para algo que faça mais sentido para a aplicação. & 1. Problema cosmético ou superficial \\\hline
    \#18 & Várias páginas & Movimentação através do site somente pelo teclado fica difícil pela cor azulada da maior parte dos botões, dessa forma o highlight não é facilmente visível e é quase impossível saber o que está selecionado & 1. Problema cosmético ou superficial \\\hline
    
\end{longtable}
\clearpage








% ----------------TABELA AVALIACAO CONSOLIDADA PT2

\subsection{Tabela 2: Problemas e Soluções Propostas}
\label{tab:app:avCons2}

\begin{longtable}{p{0.5cm} p{0.45\textwidth} p{0.45\textwidth}} 
    
    \renewcommand{\arraystretch}{1.5} 
    \textbf{ID} & \textbf{Descrição do Problema} & \textbf{Sugestões de Melhorias/Correções} \\\hline


    % --- Corpo da Tabela 2 ---
    \#1 & Fluxo de telas muito complicado. As telas intermediárias que se tornam redundantes & remover telas intermediárias desnecessárias \\\hline
    \#2 & Redirecionamento excessivo para features que podiam ser implementadas internamente & É importante estruturar o site para modularizar as funcionalidades necessárias. [...] O objetivo é evitar projetos monolíticos, que tendem a ficar mal organizados e com baixa otimização. \\\hline
    \#3 & Falta de opção de filtro para páginas que mostram muitas informações & Adicionar formas personalizadas para o usuário filtrar os horários através de interfaces interativas. (PARCIALMENTE RESOLVIDO) \\\hline
    \#4 & Descrições alternativas (alt) redundantes e muito genéricas para ícones e imagens & Colocar descrições adequadas para cada ícone, relacionando com seu significado e que auxiliem na navegação do site. \\\hline
    \#5 & ícones na barra inferior não possuem legenda (ex: casa, interrogação, envelope, pessoa). O usuário precisa clicar no ícone para identificar o que este representa. & Adicionar legendas descritivas e sugestivas diretamente abaixo dos ícones e/ou adicionar um "tooltip", ou seja, ao usuário passar o mouse por cima do botão um pequeno diálogo aparece dando mais informações e contexto ao ícone. (RESOLVIDO) \\\hline
    \#6 & A página não tem um main landmark – ou seja, não existe um ponto semanticamente definido que represente o conteúdo principal da página, sendo um problema para a identificação das partes do site pelos leitores de tela & Adicionar as demarcações corretas para landmark no código fonte. \\\hline
    \#7 & Algumas páginas dedicadas a guiar o usuário não contem nenhuma informação & Remover páginas vazias ou adicionar informações relevantes (RESOLVIDO) \\\hline
    \#8 & botões formatados como links. Deixa o layout confuso e simplifica demais o layout, levando o usuário a pensar que os botões são somente texto não reativo & Reformular tela para deixar botões mais similares a botões e deixar texto mais visível (PARCIALMENTE RESOLVIDO) \\\hline
    \#9 & Símbolos do cardápio (legenda) não indicam significado de forma explícita & adicionar alguma forma de descrição da legenda, mesmo que opcional \\\hline
    \#10 & Falta de botões e opções facilmente acessíveis para desfazer ações do usuário & Adicionar botões de retorno \\\hline
    \#11 & Não existe a opção de costumizar o site para modo escuro/claro & Inicialmente adicionar um botão na home/tela de configuração a fim de alterar entre tema escuro e claro. Futuramente, adicionar opções mais personalizadas como alterar brilho, constraste e saturação \\\hline
    \#12 & O usuário é deslogado imediatamente ao sair da aplicação & Adicionar um token de login (ex: JWT token) para que demore ao menos algumas horas para expirar ou opção de "permanecer logado" a fim de otimizar o tempo do usuário \\\hline
    \#13 & Textos em algumas páginas têm contraste baixo com o fundo & Reforçar design para melhorar os valores de cor, evitando valores muito próximos e pesos de fontes que são muito finos \\\hline
    \#14 & Quando se acessa a localização de algo, ele redireciona para o google maps, porém usa coordenadas ao invés de endereço. & Trocar coordenadas pelo endereço de local (RESOLVIDO) \\\hline
    \#15 & Em Ajuda, as opções de Carteirinha e Configurações podem causar confusão por parecerem abas de interação com o sistema. Carteirinha pode parecer que vai redirecionar para carteirinha (como na home) e Configurações para uma página de configurações do Aplicativo (cuja qual nem existe). & mudar ícones ou formato dos botões para deixar mais adequado à mensagem que a página precisa passar (RESOLVIDO) \\\hline
    \#16 & Após fazer o login, o usuário é sempre redirecionado para a tela home e não a tela que queria acessar & Mudar sistema de router do login \\\hline
    \#17 & ícones desproporcionalmente grandes. Ao aplicar zoom o hud é mal comportado, não mantendo um mesmo padrão ou se adaptando para algo que faça mais sentido para a aplicação. & Refatorar sistema de redimensionamento (PARCIALMENTE RESOLVIDO) \\\hline
    \#18 & Movimentação através do site somente pelo teclado fica difícil pela cor azulada da maior parte dos botões, dessa forma o highlight não é facilmente visível e é quase impossível saber o que está selecionado & Modificar o CSS para indicar melhor quando um botão está selecionado \\\hline
    
\end{longtable}

\end{document}
