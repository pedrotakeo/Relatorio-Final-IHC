\documentclass{article}
\usepackage[portuguese]{babel}
\usepackage{graphicx} 
\usepackage{longtable}

\title{Relatório Final IHC}
\author{
    Beatriz Pontes Camargo\\
    \footnotesize (GRR20242966)\\[1em]
    Eduardo Kaluf\\
    \footnotesize (GRR202421770)\\[1em]
    Giuliano Tavares\\
    \footnotesize (GRR202420305)\\[1em]
    \and 
    Pedro Takeo Shima\\
    \footnotesize (GRR20240627)\\[1em]
    Sergio Sivonei de Sant'Ana Filho\\
    \footnotesize (GRR20242337)\\[1em]
    Thamiris Yamate Fischer\\
    \footnotesize (GRR20243604)\\[1em]
}

\begin{document}
\maketitle

\section{Ficha de Identificação}

Lorem ipsum dolor sit amet consectetur adipiscing elit. Quisque faucibus ex sapien vitae pellentesque sem placerat. In id cursus mi pretium tellus duis convallis. Tempus leo eu aenean sed diam urna tempor. Pulvinar vivamus fringilla lacus nec metus bibendum egestas. Iaculis massa nisl malesuada lacinia integer nunc posuere. Ut hendrerit semper vel class aptent taciti sociosqu. Ad litora torquent per conubia nostra inceptos himenaeos.

\section{Matriz de avaliação}

Lorem ipsum dolor sit amet consectetur adipiscing elit. Quisque faucibus ex sapien vitae pellentesque sem placerat. In id cursus mi pretium tellus duis convallis. Tempus leo eu aenean sed diam urna tempor. Pulvinar vivamus fringilla lacus nec metus bibendum egestas. Iaculis massa nisl malesuada lacinia integer nunc posuere. Ut hendrerit semper vel class aptent taciti sociosqu. Ad litora torquent per conubia nostra inceptos himenaeos.

\section{Descrição dos problemas, severidade e sugestões de melhoria}

Lorem ipsum dolor sit amet consectetur adipiscing elit. Quisque faucibus ex sapien vitae pellentesque sem placerat. In id cursus mi pretium tellus duis convallis. Tempus leo eu aenean sed diam urna tempor. Pulvinar vivamus fringilla lacus nec metus bibendum egestas. Iaculis massa nisl malesuada lacinia integer nunc posuere. Ut hendrerit semper vel class aptent taciti sociosqu. Ad litora torquent per conubia nostra inceptos himenaeos.

\section{Apreciação geral da qualidade do sistema com base nos resultados das avaliações}

Lorem ipsum dolor sit amet consectetur adipiscing elit. Quisque faucibus ex sapien vitae pellentesque sem placerat. In id cursus mi pretium tellus duis convallis. Tempus leo eu aenean sed diam urna tempor. Pulvinar vivamus fringilla lacus nec metus bibendum egestas. Iaculis massa nisl malesuada lacinia integer nunc posuere. Ut hendrerit semper vel class aptent taciti sociosqu. Ad litora torquent per conubia nostra inceptos himenaeos.

\section{Apreciação sobre os resultados da avaliação}

Lorem ipsum dolor sit amet consectetur adipiscing elit. Quisque faucibus ex sapien vitae pellentesque sem placerat. In id cursus mi pretium tellus duis convallis. Tempus leo eu aenean sed diam urna tempor. Pulvinar vivamus fringilla lacus nec metus bibendum egestas. Iaculis massa nisl malesuada lacinia integer nunc posuere. Ut hendrerit semper vel class aptent taciti sociosqu. Ad litora torquent per conubia nostra inceptos himenaeos.

\clearpage

\appendix

\section*{APÊNDICES}

\section{Matriz de Avaliação}

\begin{longtable}{l p{0.6\textwidth} l}
    \centering 
    \renewcommand{\arraystretch}{1.5} 

    \caption{Matriz de Avaliação de Usabilidade} \\ 
    \toprule
    \textbf{ID} & \textbf{Problema} & \textbf{Sev} \\
    \midrule
    \endfirsthead 
    
    \toprule
    \multicolumn{3}{c}{Continuação da Matriz de Avaliação} \\[2em]
    \textbf{ID} & \textbf{Problema} & \textbf{Sev} \\
    \midrule
    \endhead 
    \multicolumn{3}{r}{\footnotesize Continua na próxima página...} \\
    \endfoot 

    \bottomrule
    \endlastfoot 
    
    \#01 & fluxo de telas muito complicado. telas intermediárias que se tornam redundantes & 3. Problema grande/grave \\
        \#02 & horários do intercampi são mostrados simultaneamente e não apresentam alguma forma de filtro & 3. Problema grande/grave \\
        \#03 & Muitas descrições alternativas são idênticas no site, apesar dos itens serem completamente diferentes & 3. Problema grande/grave \\
        \#04 & ícones na barra inferior não possuem legenda (ex: casa, intergração, envelope, pessoa). O usuário precisa clicar no ícone para identificar o que este representa. & 3. Problema grande/grave \\
        \#05 & Descrição dos ícones é a mesma "icone" havendo redundância. Pode gerar dúvidas para quem utiliza leitores de tela & 3. Problema grande/grave \\
        \#06 & Para acessar o vídeo de como preencher um comentário de sugestão é exaustivo e pouco óbvio. Ou seja, o link do vídeo, para um usuário desatento é pouco perceptível, além de demandar grande esforço de procurar e aprender um tutorial para poder deixar uma sugestão & 3. Problema grande/grave \\
        \#07 & A página não tem um main landmark – ou seja, não existe um ponto arquitetônico definido que represente o conteúdo principal da página, sendo um problema para a identificação das partes do site pelos leitores de tela & 3. Problema grande/grave \\
        \#08 & Redirecionamento excessivo para features que podiam ser implementadas internamente & 3. Problema grande/grave \\
        \#09 & páginas que precisam de login do usuário precisam passar por duas páginas antes de realmente levar para o login. & 2. Problema médio \\
        \#10 & algumas páginas dedicadas a guião não contem nenhuma informação & 2. Problema médio \\
        \#11 & botões formatados como links. Deixa o layout confuso e simplifica demais o layout, levando o usuário a pensar que os botões são somente texto não reativo & 2. Problema médio \\
        \#12 & símbolos do cardápio (legenda) não indicam significado de forma explícita & 2. Problema médio \\
        \#13 & É necessário rolagem completa da página para achar o botão voltar nos horários do intercampi. Ações utilizadas frequentemente devem estar facilmente visíveis & 2. Problema médio \\
        \#14 & Não existe a opção de costumizar o site para modo escuro/claro & 2. Problema médio \\
        \#15 & Notícias desatualizadas e redirecionamento para página do Instagram & 2. Problema médio \\
        \#16 & O usuário é deslogado imediatamente ao sair da aplicação & 2. Problema médio \\
        \#17 & Falta de botão de "voltar" dentro das seções internas, forçando uso do navegador & 2. Problema médio \\
        \#18 & Textos de manchetes (ex: "Curitiba 332 anos!") têm contraste baixo com o fundo da imagem & 2. Problema médio \\
        \#19 & Quando se acessa a localização de algo, ele redireciona para o google maps, porém usa coordenadas ao invés de endereço. & 2. Problema médio \\
        \#20 & Em "minha conta", as informações aparecem em campos que parecem editáveis mesmo sem ser. & 2. Problema médio \\
        \#21 & Em Ajuda, as opções de Carteirinha e Configurações podem causar confusão por parecerem abas de interação com o sistema. Carteirinha pode parecer que vai redirecionar para carteirinha (como na home) e Configurações para uma página de configurações do Aplicativo (cuja qual nem existe). & 2. Problema médio \\
        \#22 & Após fazer o login, o usuário é sempre redirecionado para a tela home e não a tela que queria acessar & 1. Problema cosmético/superficial \\
        \#23 & ícones de features muito similares de forma repetida na home screen & 1. Problema cosmético/superficial \\
        \#24 & sistema de grid para escolha de campi e cardápio parece muito complicado devido a quantidade de itens & 1. Problema cosmético/superficial \\
        \#25 & ícones desproporcionalmente grandes. Ao aplicar zoom o hud é mal comportado, não mantendo uma mesmo padrão ou se adaptando para algo que faça mais sentido para a aplicação. & 1. Problema cosmético/superficial \\
        \#26 & Movimentação através do site somente pelo teclado fica difícil pela cor azulada da maior parte dos botões, dessa forma o highlight não é facilmente visível e quase impossível saber o que está selecionado & 1. Problema cosmético/superficial \\
        \#27 & ícones de ajuda (ex: RU, Intercampi, Campus) estão bem distribuídos, mas não há agrupamento visual ou hierarquia & 1. Problema cosmético/superficial \\
        \#28 & botões de voltar a página anterior não estão no padrão conforme outros botões, facilmente confundíveis com texto simples na interface. & 1. Problema cosmético/superficial \\
        \#29 & Aba Sugestões o texto é ruim de ler e abre em outra página com instruções um pouco confusas & 1. Problema cosmético/superficial \\
        \#30 & Na seção ajuda a qual possui uma grade de ícone mostrando os tópicos de ações disponíveis, o design feito foi usando um ícone para simbolizar o tópico e abaixo do ícone um texto dizendo o nome do tópico, porém ao entrar em algum dos tópicos o design muda e é optado apenas por deixar o título, abandonando o ícone & 1. Problema cosmético/superficial \\
        \#31 & A galeria de fotos estão com fotos dos campi muito grandes, ocupando um espaço desnecessário na tela e sendo necessário mais 'scrolling' para ver os outros campi da universidade, dificultando a comunicabilidade. Além do mais, o nome dos campi estão pequenos em comparação com as fotos, levando o usuário a ficar um pouco perdido durante o primeiro acesso caso não conheça o campus olhando apenas a foto. Outro problema é a qualidade das fotos está com uma resolução baixa. & 1. Problema cosmético/superficial \\
        \#32 & falta de informações que seriam relevantes no contexto do aplicativo, como localização de bibliotecas e diferentes setores. & 0. Não é necessariamente um problema \\
    
\end{longtable}

\end{document}
